% Begin

%%%%%%%%%%%%%%%%%%%%%%%%%%%%%%%%%%%%%
\chapter*{I.}
\addcontentsline{toc}{chapter}{I.}
\label{ch:1}
\section*{}

\vskip .3cm
{
Définition {\bf I.1}. --- \it Soient $S$ un ensemble fini, $A$ une partie de $\mathfrak{P}_2(S)$, mais $A$ différent de $\mathfrak{P}_2(S)$, $F$ une partie de $\mathfrak{P}_3(S)$ satisfaisant les conditions suivantes~:
}
\vskip .3cm

\vskip .3cm
{
Proposition {\bf I.2}. --- \it 
}
\vskip .3cm
{\bf Démonstration}.






\vskip .3cm
{
Théorème {\bf I.3}. --- \it L'icosaèdre est un polyèdre combinatoire, régulier et tous les icosaèdres sont isomorphes. 
}
\vskip .3cm
{\bf Démonstration}.





\vskip .3cm
{
Proposition {\bf I.4}. --- \it Soit $\pi = (S, A, F, R, R', R'')$ un icosaèdre alors
\begin{itemize}
    \item[i)] card~$S = 12$
    \item[ii)] card~$A = 30$
    \item[iii)] card~$F =20$
    \item[iv)] card~$\Rep (\pi) =$ card~$\Aut(\pi) = 120$
\end{itemize}
}
\vskip .3cm
{\bf Démonstration}.

% End
