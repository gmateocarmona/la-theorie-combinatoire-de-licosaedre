% Begin

%%%%%%%%%%%%%%%%%%%%%%%%%%%%%%%%%%%%%%%%%%%%%%%%%%%%%%%%%%%%%%%
\chapter*{LA THÉORIE COMBINATOIRE DE L'ICOSAÈDRE}\thispagestyle{empty}
\section*{}

%%%%%%%%%%%%%%%%%%%%%%%
\addcontentsline{toc}{section}{I.}

{
Définition I.1. --- \it Soient $S$ un ensemble fini, $A$ une partie de $\mathfrak{P}_2(S)$, mais $A$ différent de $\mathfrak{P}_2(S)$, $F$ une partie de $\mathfrak{P}_3(S)$ satisfaisant les conditions suivantes~:
\begin{enumerate}
    \item[I] $\forall s \in S \quad \operatorname{card}\{ s' \in S | \{s, s' \} \in A \} = 5$,
    \item[II] $\forall a \in A \quad \operatorname{card}\{ f \in F | a \subset f \} = 2$,
    \item[III] soient $f \in F, s, s' \in f \quad \Rightarrow \quad \{s, s'\} \in A$,
    \item[IV] soient $ s,s' \in S \quad \Rightarrow \quad \exists$ une suite $s = s_0, s_1, \dots, s_n = s'$ dans $S$ telle que $\{s_i, s_{i - 1}\} \in A \ \ \forall i \in \{i,\dots,n \}$.
\end{enumerate}
Les relations d'incidence $R, R', R''$ sont définies par les relations d'inclusion.

On appelle $\Pi = (S, A, F, R, R', R'')$ un \textup{icosaèdre}.
}
\vskip .3cm
{
Proposition I.2. --- \it L'icosaèdre est un polyèdre combinatoire, connexe.\footnote{Référence~: certificat de stage de Bernard Roys}     
}
\vskip .3cm
Démonstration. --- 
\begin{enumerate}
    \item $\forall a \in A$ est condition que $\operatorname{card}\{ s \in S \mid (s,a) \in R' \} = 2$ car $A \subset \mathcal{P}_2(S)$
    \item $\forall a \in A$ est condition que $\operatorname{card}\{ f \in F \mid (a,f) \in R'' \} = 2$ à cause de II
    \item $\forall (s,f) \in R$ est condition que $\operatorname{card}\{ a \in A \mid (s,a) \in R' \land (a,f) \in R'' \} = 2$ à cause de III
    \item Soit $f \in F$ alors $\bigl( s(f),\ A(f),\ R''(f) \bigr)$ est un triangle donc un polygone combinatoire
    \item Soit $s \in S$ alors $\bigl( A(s),\ F(s),\ R''(s) \bigr)$ a la structure d'un pentagone.
\end{enumerate}
\vskip .3cm
{
Théorème I.3. --- \it L'icosaèdre est un polyèdre combinatoire, régulier et tous les icosaèdres sont isomorphes.
}
\vskip .3cm
Démonstration. --- Soient $\Pi, \Pi'$ deux icosaèdres, $r = (s,a,f)$ et $r' = (s',a',f')$ et $\operatorname{Rep}(\Pi\Pi')$. On va démontrer qu'il y a un unique isomorphisme $\varphi$ de polyèdres combinatoires de $\Pi$ sur $\Pi'$, tel que $\varphi(r) = r'$.

Ce qui montre que tous les icosaèdres sont isomorphes et par $\Pi \cong \Pi'$ ça montre que $\Pi$ est un polyèdre régulier (i.e. le groupe $G = \operatorname{Aut}(\Pi)$ des automorphismes de $\Pi$ opère transitivement sur $E = \operatorname{Rep}(\Pi)$).

Le repère $r = (s, a, f)$ détermine un pentagone $P_s$ autour du sommet $s$~:

%%%%

Pour chaque arête $ai$, il y a une deuxième face $gi$ qui est incidente à $ai$. Soit $ti$ la troisième sommet de $gi$ qui n'est pas incident à $ai$.

%%%%

On trouve que $s_i \neq t_j \quad \forall (i,j)$, car soit $s_i = t_j$ pour un $(i,j)$ nécessairement $i \neq j$, de plus on trouve que $s_i = t_i \quad \forall i \in \{1,\dots,5\}$, mais dans ce cas-là il y a $A \neq \mathcal{P}_2(S)$, ce qu'on a exclu.

Comme il y a une structure pentagonale autour de chaque sommet $s_i$, l'ensemble $\{t_j, t_{j+1}\}$ forme une arête $\forall j \in \{1,\dots,5\}$, de même l'ensemble $\{t_5, t_1\}$ forme une arête.

Pour chaque $t_i$ il y a encore une seule cinquième arête $b_i$ qui est incidente à $t_i$. Donc tous les $t_i$ sont différents l'un de l'autre et toutes les arêtes $b_i$ sont incidentes au même sommet $\{ \infty \}$.








\vskip .3cm
{\bf I.2.}

\vskip .3cm
{\bf I.3.}

\vskip .3cm
{\bf I.4.}

%%%%%%%%%%%%%%%%%%%%%%%
\subsection*{II. Distance combinatoire}\label{sec:2}%
\addcontentsline{toc}{section}{II. Distance combinatoire}

%%%%%%%%%%%%%%%%%%%%%%%
\subsection*{III. L'antipodisme d'un polyèdre combinatoire}\label{sec:3}%
\addcontentsline{toc}{section}{III. L'antipodisme d'un polyèdre combinatoire}

%%%%%%%%%%%%%%%%%%%%%%%
\subsection*{IV. Définition de l'application $\rho$ pour l'icosaèdre}\label{sec:4}%
\addcontentsline{toc}{section}{IV. Définition de l'application $\rho$ pour l'icosaèdre}

%%%%%%%%%%%%%%%%%%%%%%%
\subsection*{V. Orientation d'un ensemble fini}\label{sec:5}%
\addcontentsline{toc}{section}{V. Orientation d'un ensemble fini}

%%%%%%%%%%%%%%%%%%%%%%%
\subsection*{VI. Construction d'un icosaèdre en termes d'un ensemble à 5 éléments}\label{sec:6}%
\addcontentsline{toc}{section}{VI. Construction d'un icosaèdre en termes d'un ensemble à 5 éléments}






% End
