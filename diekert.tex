% Begin

%%%%%%%%%%%%%%%%%%%%%%%%%%%%%%%%%%%%%%%%%%%%%%%%%%%%%%%%%%%%%%%
\chapter*{LA THÉORIE COMBINATOIRE DE L'ICOSAÈDRE}\thispagestyle{empty}
\section*{}

%%%%%%%%%%%%%%%%%%%%%%%
\subsection*{I.}\label{sec:1}%
\addcontentsline{toc}{section}{I.}

\vskip .3cm
{\bf I.1.}



\vskip .3cm
{\bf I.2.}

\vskip .3cm
{\bf I.3.}

\vskip .3cm
{\bf I.4.}

%%%%%%%%%%%%%%%%%%%%%%%
\subsection*{II. Distance combinatoire}\label{sec:2}%
\addcontentsline{toc}{section}{II. Distance combinatoire}

%%%%%%%%%%%%%%%%%%%%%%%
\subsection*{III. L'antipodisme d'un polyèdre combinatoire}\label{sec:3}%
\addcontentsline{toc}{section}{III. L'antipodisme d'un polyèdre combinatoire}

%%%%%%%%%%%%%%%%%%%%%%%
\subsection*{IV. Définition de l'application $\rho$ pour l'icosaèdre}\label{sec:4}%
\addcontentsline{toc}{section}{IV. Définition de l'application $\rho$ pour l'icosaèdre}

%%%%%%%%%%%%%%%%%%%%%%%
\subsection*{V. Orientation d'un ensemble fini}\label{sec:5}%
\addcontentsline{toc}{section}{V. Orientation d'un ensemble fini}

%%%%%%%%%%%%%%%%%%%%%%%
\subsection*{VI. Construction d'un icosaèdre en termes d'un ensemble à 5 éléments}\label{sec:6}%
\addcontentsline{toc}{section}{VI. Construction d'un icosaèdre en termes d'un ensemble à 5 éléments}



% End
